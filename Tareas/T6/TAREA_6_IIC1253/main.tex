% Plantilla para documentos LaTeX para enunciados
% Por Pedro Pablo Aste Kompen - ppaste@uc.cl
% Licencia Creative Commons BY-NC-SA 3.0
% http://creativecommons.org/licenses/by-nc-sa/3.0/

\documentclass[12pt]{article}

% Margen de 1 pulgada por lado
\usepackage{fullpage}
% Incluye gráficas
\usepackage{graphicx}
% Packages para matemáticas, por la American Mathematical Society
\usepackage{amssymb}
\usepackage{amsmath}
% Desactivar hyphenation
\usepackage[none]{hyphenat}
% Saltar entre párrafos - sin sangrías
\usepackage{parskip}
% Español y UTF-8
\usepackage[spanish]{babel}
\usepackage[utf8]{inputenc}
% Links en el documento
\usepackage{hyperref}
\usepackage{fancyhdr}
\setlength{\headheight}{15.2pt}
\setlength{\headsep}{5pt}
\pagestyle{fancy}
% Código en el documento
\usepackage{listings}

\newcommand{\N}{\mathbb{N}}
\newcommand{\Exp}[1]{\mathcal{E}_{#1}}
\newcommand{\List}[1]{\mathcal{L}_{#1}}
\newcommand{\EN}{\Exp{\N}}
\newcommand{\LN}{\List{\N}}

\newcommand{\comment}[1]{}
\newcommand{\lb}{\\~\\}
\newcommand{\eop}{_{\square}}
\newcommand{\hsig}{\hat{\sigma}}
\newcommand{\ra}{\rightarrow}
\newcommand{\lra}{\leftrightarrow}

% Cambiar por nombre completo + número de alumno
\newcommand{\alumno}{Alvaro Panozo - 24664}
\rhead{Tarea N - \alumno}

\begin{document}
\thispagestyle{empty}
% Membrete
% PUC-ING-DCC-IIC1103
\begin{minipage}{2.3cm}
\includegraphics[width=2cm]{img/logo.pdf}
\vspace{0.5cm} % Altura de la corona del logo, así el texto queda alineado verticalmente con el círculo del logo.
\end{minipage}
\begin{minipage}{\linewidth}
\textsc{\raggedright \footnotesize
Pontificia Universidad Católica de Chile \\
Departamento de Ciencia de la Computación \\
IIC1253 - Matemáticas Discretas \\}
\end{minipage}


% Titulo
\begin{center}
\vspace{0.5cm}
{\huge\bf Tarea 6}\\
\vspace{0.2cm}
\today\\
\vspace{0.2cm}
\footnotesize{2º semestre 2025 - Profesores M. Arenas - A. Kozachinskiy - M. Romero}\\
\vspace{0.2cm}
\footnotesize{\alumno}
\rule{\textwidth}{0.05mm}
\end{center}



\section*{Respuestas}
% Estas numeracion es solo de ejemplo

\subsection*{Pregunta 1}
\subsubsection*{Pregunta 1.1}
$fof$ es inyectiva, entonces: \newline
$\forall a,b \in A: f(f(a))= f(f(b)) \rightarrow a=b$\newline
Tomemos $x,y \in A$ tal que $f(x) = f(y)$, luego como $fof$ es inyectiva, aplicamos $f$ en ambos lados $f(f(x)) = f(f(y))$, y entonces $x=y$. En general, $\forall x,y\in A: f(x) = f(y) \rightarrow x=y$

\newpage
\subsubsection*{Pregunta 1.2}
Sabemos que $fof$ es sobre, entonces \newline
$\forall b\in A \exists x \in A: f(f(x)) =b$ (Notemos que $f(x): A \rightarrow A$, entonces $f(x)\in A$) Luego, decimos $a =f(x)$ tal que $f(a) =b$, entonces $\forall b \in A \exists a\in A: f(a) = b$. F es sobre
\newpage
\subsubsection*{Pregunta 1.3}
$f$ es iny. $\leftrightarrow \forall X_1,X_2 \subseteq A: f(X_1\cap X_2)=f(X_1)\cap f(X_2)$\newline
$\rightarrow$:\newline
$f(X_1\cap X_2)= \{f(a)\in B| a\in X_1\cap X_2\}$ ( notemos que $X_1 =X_2$ se cumple trivialmente )\newline
Sea $y\in f(X_1\cap X_2)$, entonces debe existir $a\in X_1\cap X_2: f(a) = y$. Sabemos que $a \in X_1$, osea $y\in f(X_1)$. Analogamente, $a\in X_2$, por lo que $y \in f(X_2)$, por lo que $y\in f(X_1)\cap f(X_2)$. \newline Por lo que $f(X_1\cap X_2) \subseteq f(X_1)\cap f(X_2)$
\newline
Ahora, sea $y\in f(X_1)\cap f(X_2)$. Entonces $y \in f(X_1) \land y\in f(X_2)$, osea $\exists a_1 \in X_1: f(a_1) =y$, y analogamente $\exists a_2 \in X_2: f(a_2)=y $. Como $f$ es iny., $a_1=a_2=a$, osea $a\in X_1\cap X_2$, y entonces $y\in f(X_1\cap X_2)$.\newline
Finalmente, $f(X_1)\cap f(X_2) \subseteq f(X_1\cap X_2)$. Se concluye que $f(X_1\cap X_2)= f(X_1)\cap f(X_2)$\newline
$\leftarrow$:\newline
Sabemos que $f(X_1)\cap f(X_2) = f(X_1\cap X_2)$. Sea $a_1,a_2$ tales que $a_1 \neq a_2$ pero $f(a_1) =f(a_2) = y$. Si tomamos $X_1 = \{a_1\}$, $X_2=\{a_2\}$, nos queda que $X_1\cap X_2 = \emptyset \rightarrow f(X_1) \cap f(X_2) =f(\emptyset) = \emptyset$. Luego, $f(X_1) \cap f(X_2)= \{f(a_1)\}\cap \{f(a_2)\}$, como $f(a_1) = f(a_2)$, $f(X_1) \cap f(X_2) = \{f(a_1)\}$. Uniendo, nos queda $\{f(a_1)\}= \emptyset$, lo que es una contradiccion que viene de asumir $a_1 \neq a_2 \land f(a_1) = f(a_2)$. Entonces, $f$ es inyectiva.\newline
\newpage
\subsubsection*{Pregunta 1.4}
$f$ es sobreyectiva $\leftrightarrow \forall Y\subseteq B: f(f^{-1}(Y))=Y$\newline
$\rightarrow$\newline
Sea $b \in f(f^{-1}(Y))$, entonces $\exists a\in f^{-1}(Y)$ tq $f(a) = b$. Como $a \in f^{-1}(Y)$, entonces $f(a) \in Y$, osea que $b \in Y$. Entonces $f(f^{-1}(Y)) \subseteq Y$\newline
Sea $b \in Y$. Como la función es sobreyectiva, toda imagen debe tener preimagen, osea $\exists a \in A$ $\forall b \in B: f(a) = b$. Entonces, $a\in f^{-1}(Y)$. Si tomamos $ f(f^{-1}(Y)) = \{f(a) \in B | a\in f^{-1}(Y)\}$, como $b=f(a)\in Y\subseteq B$, y ademas $a\in f^{-1}(Y)$, entonces $b\in f(f^{-1}(Y))$. Entonces $Y \subseteq f(f^{-1}(Y))$.\newline
Se concluye que $Y = f(f^{-1}(Y))$\newline
\newline
$\leftarrow$\newline
Como $Y$ ppuede ser cualquier subconjunto, ppodemos tomar $Y=B$. Luego, por hipótesis, $ f(f^{-1}(B)) = B$. Notemos que $f^{-1}(B)= \{a\in A| f(a) \in B\}$. Como $f$ es una funcion bien definida, $\forall a \in A \exists! b\in B$ tq $f(a) =b $. Por lo tanto $f^{-1}(B) = A$. Nos queda $f(A) = B$, que es una forma de definir sobreyectividad.
\end{document}